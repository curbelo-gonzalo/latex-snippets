% =============================================================================
% DESCRIPTION:
% This file contains commands to insert special underlines (e.g. fixed in
%   length) below text.

% =============================================================================
% PREAMBLE:
% \documentclass{article}
\usepackage{stackengine}

% =============================================================================
% REMARKS:
% This was only tested in documentclass "article". It may or may not work with
%   other classes.

% =============================================================================
% CONTENTS:

% \FLUnderline
% Description: Inserts a fixed-length straight line below a given text.
% Arguments:
%   #1 = (Optional) Text alignment. Defaults to "l".
%   #2 = (Required) Line length. Must be defined in terms of spacing commands.
%   #3 = (Required) Contents. May be empty.
% Remarks: None.
\newcommand{\FLUnderline}[3][l]{%
    \stackengine{0pt} % stacking length
                {\underline{#2\vphantom{#3}}} % stack anchor
                {#3}  % item stacked relative to the anchor
                {O}   % O = normal/overstacked
                {#1}  % alignment (l, c, r)
                {F}   % F = print the resulting stack
                {F}   % F = do NOT take the width of the whole stack is as the
                      %     width of the anchor
                {L}   % L = request a long stack
}

% \MLUnderline
% Description: Inserts a minumum-length straight line below a given text.
% Arguments:
%   #1 = (Optional) Text alignment. Defaults to "l".
%   #2 = (Required) Line length. Must be defined in terms of spacing commands.
%   #3 = (Required) Contents. May be empty.
% Remarks: None.
\newcommand{\MLUnderline}[3][l]{%
    \underline{%
        \stackengine{0pt} % stacking length
                    {#2}  % stack anchor
                    {#3}  % item stacked relative to the anchor
                    {O}   % O = normal/overstacked
                    {#1}  % alignment (l, c, r)
                    {F}   % F = print the resulting stack
                    {F}   % F = do NOT take the width of the whole stack is as
                          %     the width of the anchor
                    {L}%    L = request a long stack
    }%
}
